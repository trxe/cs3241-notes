\documentclass{beamer}
\usepackage[utf8]{inputenc}
\usepackage[T1]{fontenc}

\usetheme{Arguelles}

\title{Tutorial 1}
\subtitle{CS3241 Computer Graphics (AY22/23)}
\date{\today}
\author{Wong Pei Xian}
\institute[]{\email{e0389023@u.nus.edu}}

\begin{document}

\frame[plain]{\titlepage}

\section{Question 1}

\begin{frame}
    \frametitle{Question 1}
    To be able to display \textbf{realistic} images, our display devices need to be able to produce 
    every frequency in the visible light spectrum.
    
    \vspace*{1em}
    
    True or false? Why? What are the advantages and disadvantages?
\end{frame}

\begin{frame}
    \frametitle{Three-Color Theory}

    To be \textbf{realistic to human} \\
    $\Rightarrow$ To be compatible with 
    \textcolor{red}{human} \textcolor{blue}{visual} \textcolor{green}{system}
    (L01, slide 35)

    \begin{itemize}
        \item Rods: Monochromatic
        \item \textbf{Cones}: Color sensitive to wavelengths
        \begin{itemize}
            \item Long \textcolor{red}{$\approx$ red}
            \item Medium \textcolor{green}{$\approx$ green}
            \item Short \textcolor{blue}{$\approx$ blue}
        \end{itemize}
    \end{itemize}

    \textbf{Proportion} of the three gives us the sensation of different colors.
\end{frame}

\begin{frame}
    \frametitle{Cones sensitivity}

\end{frame}

\begin{frame}
    \frametitle{Additive color}

    

\end{frame}

\section{Question 2}

\begin{frame}
    \frametitle{Question 2}

    Each pixel in a frame-buffer has 8 bits for each of the R, G and B channels. How many different
    colors can each pixel represent? Is this enough? On some systems, each pixel has only 8 bits
    (for all R, G, and B combined). How would you allocate the bits to the R, G and B primaries?

\end{frame}

\begin{frame}
    \frametitle{8-bit representation of color}

    

\end{frame}

\section{Question 3}

\begin{frame}
    \frametitle{Question 3}
    Referring to Lecture 1 Slide 26. If an imaginary image plane is d unit distance in front of the
    pinhole camera, what are the coordinates of the projection (on the imaginary image plane) of
    the 3D point (x, y, z)?
\end{frame}

\section{Question 4}

\begin{frame}
    \frametitle{Question 4}
    Why do we need a primitive assembly stage in the rendering pipeline architecture?
\end{frame}

\section{Question 5}

\begin{frame}
    \frametitle{Question 5}
    What does the rasterization stage (rasterizer) do in the rendering pipeline architecture?
    Describe what it does to a triangle that is supposed to be filled, and the three vertices have
    different color. Assume smooth shading is turned on.
\end{frame}

\section{Question 6}

\begin{frame}
    \frametitle{Question 6}
    What is hidden-surface removal? When is it not necessary?
\end{frame}

\section{Question 7}

\begin{frame}
    \frametitle{Question 7}
    Which of the two following program fragments is more efficient? Why?
    Can the same optimization be done for the case of \texttt{GL\_POLYGON}?
\end{frame}

\section{Question 8}

\begin{frame}
    \frametitle{Question 8}
    OpenGL supports the GL\_TRIANGLES primitive type. Why do you think that OpenGL also
    supports GL\_TRIANGLE\_FAN and GL\_TRIANGLE\_STRIP?
\end{frame}

\section{Question 9}

\begin{frame}
    \frametitle{Question 9}
    Devise a test to check whether a polygon in 3D space is planar.
\end{frame}

\section{Question 10}

\begin{frame}
    \frametitle{Question 10}

    Devise a test to check whether a polygon on the x-y plane is convex.
\end{frame}

\end{document}