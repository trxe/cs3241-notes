\documentclass{beamer}
\usepackage[utf8]{inputenc}
\usepackage[T1]{fontenc}
\usepackage{graphicx}
\usepackage{tcolorbox}
\usepackage{hyperref}
\hypersetup{
    colorlinks=true,
    linkcolor=white,
    urlcolor=cyan,
    urlbordercolor=cyan,
}
\graphicspath{ {./images/} }

\usetheme{Arguelles}

\title{Tutorial 2}
\subtitle{CS3241 Computer Graphics (AY22/23)}
\date{\today}
\author{Wong Pei Xian}
\institute[]{\email{e0389023@u.nus.edu}}

\begin{document}

\frame[plain]{\titlepage}

\section{Question 1}

\begin{frame}
    \frametitle{Question 1}
    What is a GLUT display callback function?  
    Give example events for which the display callback function should be called.
\end{frame}

\begin{frame}
    \frametitle{GLUT function}

    \begin{tcolorbox}
        \textbf{GLUT}: OpenGL Utility Toolkit (Lecture 2 slide 10), \\
        
        a \textbf{library} that provides I/O functionality common to \textcolor{teal}{all window systems}.
    \end{tcolorbox}

    \begin{center}
        \includegraphics[]{q1-glut-callbacks.png}
    \end{center}

\end{frame}

\begin{frame}
    \frametitle{GLUT display callback}

    \begin{itemize}
        \item \texttt{glutDisplayFunc()}
        \item User-defined callback to register
        \item Executed on each window refresh.
    \end{itemize}

\end{frame}

\section{Question 2}

\begin{frame}
    \frametitle{Question 2}

    What is the use of the GLUT function \texttt{glutPostRedisplay()}?

\end{frame}

\begin{frame}
    \frametitle{\texttt{glutPostRedisplay}}

    The execution of the \texttt{glutPostRedisplay()} function tells GLUT to call the 
    display callback function at the end of the current event loop. \\

\end{frame}

\begin{frame}
    \frametitle{Question 2}

    When do we want to call the \texttt{glutPostRedisplay()} function? 

\end{frame}

\begin{frame}
    \frametitle{}

    When we explicitly want the rendered image to be updated.  

    \vspace{1em}

    \begin{tcolorbox}
        Q: Why don’t we call the display callback function directly to update the image?\\
        \vspace{1em}
        A: \textbf{Multiple calls} to \texttt{glutPostRedisplay} may be made in a single iteration of main loop,
        we don't want to redisplay (reapply graphics to buffer) everytime.
    \end{tcolorbox}

    \hyperlink{Reference}{https://www.opengl.org/resources/libraries/glut/spec3/node20.html}

\end{frame}

\section{Question 3}

\begin{frame}
    \frametitle{Question 3}
    How does double buffering work?  Why do we use it?
\end{frame}

\begin{frame}
    \frametitle{Double buffering}

    \begin{center}
        \includegraphics[scale=0.4]{q3-step1.png}
    \end{center}

    \begin{itemize}
        \item Back buffer: apply graphics \textbf{WHILE}
        \item Front buffer: display graphics 
    \end{itemize}

\end{frame}

\begin{frame}
    \frametitle{Double buffering}

    \begin{center}
        \includegraphics[scale=0.4]{q3-step2.png}
    \end{center}

    Swapping is fast and seamless.

\end{frame}

\begin{frame}
    \frametitle{Prevents screen tearing}

    \begin{center}
        \includegraphics[scale=0.3]{screen-tear.jpg}
    \end{center}

    \begin{tcolorbox}
        \textbf{Screen tearing}: 
        when the rate of graphics feed application $\neq$ window refresh rate.
    \end{tcolorbox}

    Notice how double buffering solves this by making sure graphics are not applied to the currently displayed frame,
    only swapping the frames when the application is complete.

\end{frame}

\section{Question 4}

\begin{frame}
    \frametitle{Question 4}
    The use of any special hidden surface removal method is not necessary if we can sort the polygons in a back-to-front order 
    and render these polygons in that order. (Tutorial 1 Q6)\\

    \vspace{1em}
    
    Is it \textbf{always possible} that any set of polygons can be sorted in a back-to-front order?
\end{frame}

\begin{frame}
    \frametitle{Cyclic overlap}

    \begin{center}
        \includegraphics[scale=0.6]{cyclic-overlap.png}
    \end{center}

\end{frame}

\section{Question 5}

\begin{frame}
    \frametitle{Question 5a}

    What is an OpenGL viewport?
\end{frame}

\begin{frame}
    \frametitle{Viewport}

    \begin{tcolorbox}
        OpenGL viewport: A rectangular region of the window in which OpenGL can draw.
    \end{tcolorbox}

\end{frame}

\begin{frame}
    \frametitle{Question 5b}

    How do you specify one?
\end{frame}

\begin{frame}
    \frametitle{\texttt{glViewport}}

    \begin{center}
        \includegraphics[scale=0.3]{viewport.png}
    \end{center}

    \begin{tcolorbox}
        \begin{center}
            \texttt{glViewport(GLint x, GLint y, GLsizei w, GLsizei h)}
        \end{center}
    \end{tcolorbox}

    Note: $x,y,w,h$ are in window coordintes.

\end{frame}

\begin{frame}
    \frametitle{Question 5c}

    Can we have \textbf{multiple viewports} in one window?
\end{frame}

\begin{frame}
    \frametitle{Yes!}

    \begin{center}
        \includegraphics[scale=0.3]{multiple-viewports.png}
    \end{center}

\end{frame}

\begin{frame}
    \frametitle{Yes!}

    \begin{center}
        \includegraphics[scale=0.2]{it-takes-two.jpg}
    \end{center}

\end{frame}

\begin{frame}
    \frametitle{Question 5d, 5e}

    Can a viewport be larger than the window?
    \vspace{1em}
    If yes, what will happen?
\end{frame}

\begin{frame}
    \frametitle{Yes!}

    \begin{center}
        \includegraphics[scale=0.5]{glviewport.png}
    \end{center}

    Parameter types are \texttt{GLint} for x and y coordinates, so they can be negative and go out of the screen.\\

    \vspace{1em}

    Or \texttt{width} or \texttt{height} could also exceed window size.

    \textbf{Viewport size is independent of window size.}

\end{frame}

\begin{frame}
    \frametitle{Specification example}

    \begin{center}
        \includegraphics[scale=0.4]{viewport-window.png}
    \end{center}
    
\end{frame}

\begin{frame}
    \frametitle{Question 5f}

    When you use \texttt{glClear(GL\_COLOR\_BUFFER\_BIT)},
    are you clearing the entire window or just the viewport?
\end{frame}

\begin{frame}
    \frametitle{Question 5f}

    When you use \texttt{glClear(GL\_COLOR\_BUFFER\_BIT)},
    are you clearing the entire window or just the viewport?

    \begin{tcolorbox}
        Answer: the window.
    \end{tcolorbox}
\end{frame}

\section{Question 6}

\begin{frame}
    \frametitle{Question 6}

    Assume we have the following OpenGL function calls:

    \begin{tcolorbox}
        \texttt{\\
            glViewport( u, v, w, h ); \\
            ... \\
            gluOrtho2D( x\_min, x\_max, y\_min, y\_max );\\
        }
    \end{tcolorbox}

    Find the mathematical expressions that map a point $(x, y)$ that lies within the 
    clipping rectangle to a point $(xs, ys)$ that lies within the viewport.
\end{frame}

\begin{frame}
    \frametitle{Clip space to window space}

    \begin{center}
        \includegraphics[]{clip-window.png}
    \end{center}
    
    $x_s = u + (x-x_{\min}) (\frac{w}{x_{\max} - x_{\min}})$\\
    $y_s = v + (y-y_{\min}) (\frac{h}{y_{\max} - y_{\min}})$\\

\end{frame}

\section{Question 7}

\begin{frame}
    \frametitle{Question 7a}
    In many old CRT monitors, the pixels are not square. 
    Let’s assume the pixel width-to-height aspect ratio is 4:3. 
    \vspace{1em}

    Suppose in the \textbf{camera coordinate frame}, there is a disc in the z = 0 plane, 
    centered at (100, 200, 0), and has a radius of 10. 

    You want to draw the entire disc as big as possible inside the window, 
    and it should appear circular and not oval.
    \vspace{1em}

    \begin{tcolorbox}
        If the window size is \underline{\quad\quad}, how would you set up the \textcolor{teal}{viewport} and the
        \textcolor{violet}{orthographic projection} using OpenGL?
        \begin{itemize}
            \item 600 $\times$ 300
            \item 300 $\times$ 600
            \item 300 $\times$ 320
        \end{itemize}
    \end{tcolorbox}
\end{frame}

\begin{frame}
    \frametitle{Visualize}

    \begin{center}
        \includegraphics[scale=0.4]{q7-cam-win.png}
    \end{center}

    \begin{tcolorbox}
        Consider the case where the pixels are square first.
    \end{tcolorbox}

\end{frame}

\begin{frame}
    \frametitle{Template}

    \begin{center}
        \includegraphics[scale=1.4]{q7.png}
    \end{center}

\end{frame}

\iffalse
\begin{frame}
    \frametitle{Things to note}

    \begin{itemize}
        \item \texttt{glViewport} should span whole window
        \item How to set the orthographic projection?
        \item If we set the clipping planes to \texttt{(c.x - r, c.x + r, c.y - r, c.y + r)} 
        we get an oval across the entire viewport. How do we manipulate this clipping plane to show circle instead?
    \end{itemize}

\end{frame}

\fi


\ThankYou
\begin{frame}[plain,standout]
    Thanks! Get the slides here.\\
    \vspace{2em}
    \scalebox{3}{\faGithub}\par\bigskip
    \url{https://trxe.github.io/cs3241-notes}
\end{frame}

\end{document}